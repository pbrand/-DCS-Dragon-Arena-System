\documentclass{article}
\usepackage{a4wide}



%% if your are not using LaTeX2e use instead
%% \documentstyle[bnaic]{article}

%% begin document with title, author and affiliations

\title{Report deadline1
}
\author{Patrick Brand (p.brand@student.tudelft.nl) \and
    Raies Saboerali (r.a.a.saboerali@student.tudelft.nl)}
\date{}

%support cast needs to be filled in here.

\pagestyle{empty}

\begin{document}
\maketitle
\thispagestyle{empty}

\section{Introduction}
WantGame BV wants to develop a new game engine. Because the computational complexity of these engines is high, the CTO of WantGame BV believes distributed systems offer an excellent performance - cost trade - off when supporting many concurrent users. This document discussed the analyses of the requested system.

\section{System Analysis}
\subsection{Requirements}
\subsubsection{System Requirements}
\begin{enumerate}
	\item The system must be able cope with 100 players, 20 dragons when using 5 server nodes.
	\item The system must log all game and system events.
	\item The log of all game and system events must be available on at least two server nodes of the system.
	\item The system must log all causal actions in the order in which they actually occur.
	\item The system must be able to connect and disconnect users based on data from a real workload trace taken from the Game Trace Archive.
\end{enumerate}

\subsubsection{Game Requirements}
\begin{enumerate}
	\item The game must world consists of a batlefield of 25x25 squares.
	\item Each participant occupies one single  square of the game world.
	\item Each square of the game world can host at most one particpant.
	\item There must be a xxx (e.g. 500) ms delay between two consecutive actions of the same battle participant.
	\item Each battle participant maintains its initial and remaining number of health points.
	\item Initially a player has between 10 and 20 health points and between 1 and 10 attack points.
	\item Initially a dragon has between 50 an 100 health points and between 5 and 20 attack points.
	\item A player can either move one step horizontally or vertically during a single action.
	\item A player can hit a dragon either from a distance of at most 2 horizontal or 2 vertical squares.
	\item A dragon can hit a player either from a distance of at most 2 horizontal or 2 vertical squares.
	\item A player is not able to hit another player.
	\item A player can heal another player by adding the receivers health points by an amount equal to the healer's attack points, up to the receiver's maximum initial health points. 
	\item A player can heal another player from a distance of at most 5 horizontal or 5 vertical squares.
\end{enumerate}

\section{Design choices}
\subsection{Replication}
\begin{itemize}
\item Server-initiated replicas without Permanent replicas.
\item No client-initiated replicas, cache for clients is not needed, because many updates would mean many cache updates which makes the cache obsolete.
\end{itemize}
\subsection{Consistency}
Data-centric consistency model:
\begin{itemize}
	\item The game will be server centric and not peer-to-peer.
	\item All in-game actions of the users will be sent to the server, therefore it is more efficient to compute the causal ordering of the game events on the server. 
	\item In this way the server is able to sent the ordered events to all clients. However in a peer-to-peer system, all the actions would have to be sent to all the different clients and each client would have had to compute the correct ordering of the events. 
	\item By letting the server process the incoming data of all the different clients, we take away some workload from the clients.
	\item This might even lead to inconsistensy between clients as they don't know each others result. As these results might differ per client the requirement of having at least 2 redundant copies of the game log might be impossible as it is not possible to verify which client has the righteous data.
\end{itemize} 

causal consistency:
\begin{itemize}
	\item a logical choice when taking the causal actions, described by the requirements, into consideration.
	\item Sequential would give problems at determining the order of the events. Time does not play a role here. However it actually does in our system
	\item Continuous consistency doesn't order the events at all. Causal order not guaranteed.
\end{itemize}
\newpage




\nocite{*}
\bibliographystyle{plain}
\bibliography{mybibfile}

\end{document}








